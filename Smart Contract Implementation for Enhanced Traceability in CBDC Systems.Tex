% This is a simple sample document.  For more complicated documents take a look in the exercise tab. Note that everything that comes after a % symbol is treated as comment and ignored when the code is compiled.

\documentclass[letterpaper,twocolumn,10pt]{article} % \documentclass{} is the first command in any LaTeX code.  It is used to define what kind of document you are creating such as an article or a book, and begins the document preamble

\usepackage{amsmath} % \usepackage is a command that allows you to add functionality to your LaTeX code
\usepackage[T1]{fontenc}

\title{\textbf{Smart Contract Implementation for Enhanced Traceability in Central Bank Digital Currency Systems}} % Sets article title
\author{
    Siarawit Techavanitch \\ sirawit\_tec@utcc.ac.th \\ \textit{School of Engineering} \\ University of the Thai Chamber of Commerce (UTCC) \and
    Supachate Innet \\ supachate\_inn@utcc.ac.th \\ \textit{School of Engineering} \\ University of the Thai Chamber of Commerce (UTCC) } % Sets authors name
\date{\today} % Sets date for date compiled

% The preamble ends with the command \begin{document}
\begin{document} % All begin commands must be paired with an end command somewhere
\maketitle % creates title using information in preamble (title, author, date)

% \section{Introduction} % creates a section

\textbf{\textit{Abstract}} TODO
\section{Introduction} In the present-day
Central Bank Digital Currency
concept aims to utilize the advantage point of Blockchain Technology
or Distributed Ledger Technology that provides immutable,
transparency, and security and the smart contract that plays a key feature in creating programmable money.

However, technology itself provides an advantage and eliminates the problem ideally,
but it does not seem to be practical to be done in real world and not in an efficient way to responsible for the financial crime or incidents that occur in the open network of economic.


\end{document} % This is the end of the document